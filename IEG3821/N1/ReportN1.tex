\documentclass[10pt]{article}
\usepackage{fontspec,xunicode}
\usepackage{graphicx}
\usepackage{url}
\usepackage[a4paper,hmargin=1in,vmargin=1.5in]{geometry}
\defaultfontfeatures{Mapping=tex-text}
\XeTeXlinebreaklocale "en"
\XeTeXlinebreakskip = 0pt plus 1pt
\setromanfont{Georgia}
\setmainfont{Georgia}
\setsansfont{Tahoma}
%\setmonofont{Lucida Console}
\setmonofont{Courier New}
\renewcommand{\thesection}{Task \arabic{section}:}
\renewcommand{\thesubsection}{\alph{subsection})}


\author{GUAN Hao\\05569511\\hguan5@ie.cuhk.edu.hk}
\title{IEG3821 N1:Introduction of Router Emulator and Fundamentals of IOS\\Laboratory Report}
\date{\today}
\begin{document}
\maketitle
\section{Construct the Network file of Dynagen}
According to the network sructure diagram, the modified {\tt lab1.net} is shown following.
\begin{verbatim*}
# IEG3821 – Lab1
# GUAN Hao, hguan5@ie.cuhk.edu.hk
autostart = false
ghostios = true
ghostsize = 128
mmap = true
sparsemem = true

[127.0.0.1]
workingdir = c:\temp\mydir\ieg3821\lab1 


[[ETHSW SW]]
1 = access 177
2 = access 77
3 = access 77
4 = access 77
5 = access 77
7 = access 77
8 = access 177
17 = access 177

[[FRSW FR]]
2:216 = 16:612
3:316 = 16:613
4:406 = 6:604
5:516 = 16:615

[[3660]]
idlepc = 0x60529294
image = c:\temp\MyDir\IEG3821\images\c3660-jk9o3s-mz.124-12.image
ram = 128

#### R1 ####
[[ROUTER R1]]
console = 2001
model = 3660
F0/0 = SW 1

#### R2 ####
[[ROUTER R2]]
console = 2002
model = 3660
F0/0 = SW 2
S1/0 = FR 2

#### R3 ####
[[ROUTER R3]]
console = 2003
model = 3660
F0/0 = SW 3
S1/0 = FR 3

#### R4 ####
[[ROUTER R4]]
console = 2004
model = 3660
F0/0 = SW 4
S1/0 = FR 4


#### R5 ####
[[ROUTER R5]]
console = 2005
model = 3660
F0/0 = SW 5
S1/0 = FR 5


#### R6 ####
[[ROUTER R6]]
console = 2006
model = 3660
S1/0 = FR 6
S1/1 = FR 16


#### R7 ####
[[ROUTER R7]]
console = 2007
model = 3660
F0/0 = SW 7
F1/0 = SW 17


#### R8 ####
[[ROUTER R8]]
console = 2008
model = 3660
F0/0 = SW 8

\end{verbatim*}
\section{Basic Network Configurations}
\subsection{hostname and domain name}
\begin{itemize}
\item {\bf R1}
\begin{verbatim*}
hostname R1
ip domain name ie.cuhk.edu.hk
\end{verbatim*}
\item {\bf R2}
\begin{verbatim*}
hostname R2
ip domain name ie.cuhk.edu.hk
\end{verbatim*}
\item {\bf R3}
\begin{verbatim*}
hostname R3
ip domain name ie.cuhk.edu.hk
\end{verbatim*}
\item {\bf R4}
\begin{verbatim*}
hostname R4
ip domain name ie.cuhk.edu.hk
\end{verbatim*}
\item {\bf R5}
\begin{verbatim*}
hostname R5
ip domain name ie.cuhk.edu.hk
\end{verbatim*}
\item {\bf R6}
\begin{verbatim*}
hostname R6
ip domain name ie.cuhk.edu.hk
\end{verbatim*}
\item {\bf R7}
\begin{verbatim*}
hostname R7
ip domain name ie.cuhk.edu.hk
\end{verbatim*}
\item {\bf R8}
\begin{verbatim*}
hostname R8
ip domain name ie.cuhk.edu.hk
\end{verbatim*}
\end{itemize}
\subsection{Username / Password for remote login}
\begin{itemize}
\item {\bf R1..R8}
\begin{verbatim*}
username hello password world
line vty 0 4
 login local
!
\end{verbatim*}
\end{itemize}
\subsection{IP Address on LAN interface R1, R7 and R8}
\begin{itemize}
\item {\bf R1}
\begin{verbatim*}
int f0/0
 ip addr 177.51.7.1 255.255.255.240
 no shut
!
\end{verbatim*}
\item {\bf R7}
\begin{verbatim*}
int f0/0
 ip addr 77.51.7.7 255.255.255.0
 no shut
!
int f1/0
 ip addr 177.51.7.7 255.255.255.240
 no shut
!
\end{verbatim*}
\item {\bf R8}
\begin{verbatim*}
int f0/0
 ip addr 177.51.7.8 255.255.255.240
 no shut
!
\end{verbatim*}
\item {\bf Capture}
\begin{verbatim*}
R1#sh ip route
Gateway of last resort is not set

177.51.0.0/28 is subnetted, 1 subnets
C       177.51.7.0 is directly connected, FastEthernet0/0

R7#sh ip route
Gateway of last resort is not set

77.0.0.0/24 is subnetted, 1 subnets
C       77.51.7.0 is directly connected, FastEthernet0/0
177.51.0.0/28 is subnetted, 1 subnets
C       177.51.7.0 is directly connected, FastEthernet1/0

R8#sh ip route
Gateway of last resort is not set

177.51.0.0/28 is subnetted, 1 subnets
C       177.51.7.0 is directly connected, FastEthernet0/0
\end{verbatim*}
\end{itemize}
\subsection{IP Address on loopback interfaces}
\begin{itemize}
\item {\bf R1}
\begin{verbatim*}
int lo0
 ip addr 150.51.1.1 255.255.255.0
!
\end{verbatim*}
\item {\bf R8}
\begin{verbatim*}
int lo0
 ip addr 150.51.8.8 255.255.255.0
!
\end{verbatim*}
\item {\bf Capture}
\begin{verbatim*}
R1#sh ip route
Gateway of last resort is not set

177.51.0.0/28 is subnetted, 1 subnets
C       177.51.7.0 is directly connected, FastEthernet0/0
150.51.0.0/24 is subnetted, 1 subnets
C       150.51.1.0 is directly connected, Loopback0

R8#sh ip route
Gateway of last resort is not set

177.51.0.0/28 is subnetted, 1 subnets
C       177.51.7.0 is directly connected, FastEthernet0/0
150.51.0.0/24 is subnetted, 1 subnets
C       150.51.8.0 is directly connected, Loopback0
\end{verbatim*}
The default link status of the loopback interface is {\em up}.
\end{itemize}
\subsection{Default route and Static route}
\begin{itemize}
\item {\bf R1}
\begin{verbatim*}
ip route 77.51.7.0 255.255.255.0 177.51.7.7
\end{verbatim*}
\item {\bf R7}
\begin{verbatim*}
ip route 0.0.0.0 0.0.0.0 177.51.7.0
\end{verbatim*}
\item {\bf R8}
\begin{verbatim*}
ip route 77.51.7.0 255.255.255.0 177.51.7.7
\end{verbatim*}
\item {\bf Capture}
\begin{verbatim*}
R1#sh ip route
Gateway of last resort is not set

77.0.0.0/24 is subnetted, 1 subnets
S       77.51.7.0 [1/0] via 177.51.7.7
177.51.0.0/28 is subnetted, 1 subnets
C       177.51.7.0 is directly connected, FastEthernet0/0
150.51.0.0/24 is subnetted, 1 subnets
C       150.51.1.0 is directly connected, Loopback0

R7#sh ip route
Gateway of last resort is 177.51.7.0 to network 0.0.0.0

77.0.0.0/24 is subnetted, 1 subnets
C       77.51.7.0 is directly connected, FastEthernet0/0
177.51.0.0/28 is subnetted, 1 subnets
C       177.51.7.0 is directly connected, FastEthernet1/0
S*   0.0.0.0/0 [1/0] via 177.51.7.0


R8#sh ip route
Gateway of last resort is not set

77.0.0.0/24 is subnetted, 1 subnets
S       77.51.7.0 [1/0] via 177.51.7.7
177.51.0.0/28 is subnetted, 1 subnets
C       177.51.7.0 is directly connected, FastEthernet0/0
150.51.0.0/24 is subnetted, 1 subnets
C       150.51.8.0 is directly connected, Loopback0
\end{verbatim*}
\item {\bf Verify and capture}
\begin{verbatim*}
R7#ping 150.51.1.1 source f0/0

Type escape sequence to abort.
Sending 5, 100-byte ICMP Echos to 150.51.1.1, timeout is 2 seconds:
Packet sent with a source address of 77.51.7.7
!!!!!
Success rate is 100 percent (5/5), round-trip min/avg/max = 12/71/148 ms
R7#ping 150.51.8.8 source f0/0

Type escape sequence to abort.
Sending 5, 100-byte ICMP Echos to 150.51.8.8, timeout is 2 seconds:
Packet sent with a source address of 77.51.7.7
!!!!!
Success rate is 100 percent (5/5), round-trip min/avg/max = 32/80/144 ms
\end{verbatim*}
\end{itemize}
\subsection{Configure Timezone, Date and Time, NTP Server address}
\begin{itemize}
\item {\bf R8}
\begin{verbatim*}
clock timezone HK 8
ntp master
\end{verbatim*}
\item {\bf Capture}
\begin{verbatim*}
R8#show clock
00:07:07.907 HK Sun Feb 10 2008
\end{verbatim*}
\item {\bf R1 and R7}
\begin{verbatim*}
clock timezone HK 8
ntp server 177.51.7.8
\end{verbatim*}
\item {\bf Verify and capture}
\begin{verbatim*}
R7#show clock
00:07:14.528 HK Sun Feb 10 2008
R1#show clock
00:07:29.473 HK Sun Feb 10 2008
\end{verbatim*}
\end{itemize}
\section{Configuration of DNS Server at R8}
\begin{itemize}
\item {\bf R8}
\begin{verbatim*}
ip dns server
no ip domain lookup
ip domain name ie.cuhk.edu.hk
ip host R1.ie.cuhk.edu.hk 177.51.7.1
ip host R2.ie.cuhk.edu.hk 77.51.7.2
ip host R3.ie.cuhk.edu.hk 77.51.7.3
ip host R4.ie.cuhk.edu.hk 77.51.7.4
ip host R6-4.ie.cuhk.edu.hk 66.51.46.6
ip host R6-5.ie.cuhk.edu.hk 66.51.56.6
ip host R6-23.ie.cuhk.edu.hk 66.52.236.6
ip host R77.ie.cuhk.edu.hk 77.51.7.7
ip host R177.ie.cuhk.edu.hk 177.51.7.7
ip host R8.ie.cuhk.edu.hk 177.51.7.8
\end{verbatim*}
\item {\bf R1 and R7}
\begin{verbatim*}
ip name-server 177.51.7.8
\end{verbatim*}
\item {\bf Verify and capture}
\begin{verbatim*}
R1#ping 77.51.7.7 source lo0

Type escape sequence to abort.
Sending 5, 100-byte ICMP Echos to 77.51.7.7, timeout is 2 seconds:
Packet sent with a source address of 150.51.1.1
!!!!!
Success rate is 100 percent (5/5), round-trip min/avg/max = 16/74/136 ms
R1#ping R77 source lo0

Translating "R77"...domain server (177.51.7.8) [OK]

Type escape sequence to abort.
Sending 5, 100-byte ICMP Echos to 77.51.7.7, timeout is 2 seconds:
Packet sent with a source address of 150.51.1.1
!!!!!
Success rate is 100 percent (5/5), round-trip min/avg/max = 16/100/144 ms
\end{verbatim*}
The difference between these two commands is that {\tt ping R77} need DNS server(R8) to resolve the domain name before ping packet is sent. 
\end{itemize}
\section{Configuration of DHCP Server and Client}
\subsection{Manual IP mapping to DHCP client}
\begin{itemize}
\item {\bf R7}
\begin{verbatim*}
service dhcp
ip dhcp pool R2
 host 77.51.7.2 255.255.255.0
 client-identifier 0100.0022.2222.22
 domain-name ie.cuhk.edu.hk
 dns-server 177.51.7.8
 default-router 77.51.7.7
 lease infinite
!
ip dhcp pool R3
 host 77.51.7.3 255.255.255.0
 client-identifier 0100.0033.3333.33
 domain-name ie.cuhk.edu.hk
 dns-server 177.51.7.8
 default-router 77.51.7.7
 lease infinite
!
ip dhcp pool R4
 host 77.51.7.4 255.255.255.0
 client-identifier 0100.0044.4444.44
 domain-name ie.cuhk.edu.hk
 dns-server 177.51.7.8
 default-router 77.51.7.7
 lease infinite
!
ip dhcp excluded-address 77.51.7.2 77.51.7.4
\end{verbatim*}
\item {\bf R2, R3 and R4}
\begin{verbatim*}
int f0/0
 ip addr dhcp client-id f0/0
 no shut
!
\end{verbatim*}
\end{itemize}
\subsection{Dynamic IP mapping to DHCP client}
\begin{itemize}
\item {\bf R7}
\begin{verbatim*}
ip dhcp pool R5
 network 77.51.7.0 255.255.255.0
 default-router 77.51.7.7
 dns-server 177.51.7.8
 lease infinite
!
\end{verbatim*}
\item {\bf R5}
\begin{verbatim*}
int f0/0
 ip address dhcp
 no shut
!
\end{verbatim*}
\end{itemize}
\subsection{DHCP binding at DHCP Server}
The table of DHCP binding at the DHCP server(R7).
\begin{verbatim*}
R7#show ip dhcp binding
Bindings from all pools not associated with VRF:
IP address          Client-ID/              Lease expiration        Type
                    Hardware address/
                    User name
77.51.7.1           0063.6973.636f.2d30.    Infinite                Automatic
                    3030.302e.3535.3535.
                    2e35.3535.352d.4661.
                    302f.30
77.51.7.2           0100.0022.2222.22       Infinite                Manual
77.51.7.3           0100.0033.3333.33       Infinite                Manual
77.51.7.4           0100.0044.4444.44       Infinite                Manual
\end{verbatim*}
At R5, the IP addresss obtained by DHCP is {\tt 77.51.7.1}. At the DHCP server(R7), the Client-ID of the IP address released to R5 is {\tt0063.6973.636f.2d30.3030.302e.3535.3535.2e35.3535.352d.4661.302f.30}. Decoding the Client-ID from ASCII code into TEXT, it is {\tt cisco-0000.5555.5555-Fa0/0}.
\section{Configuration of Frame Relay on Serial Interface}
\subsection{Configuration of Frame Relay on main serial interface}
\begin{itemize}
\item {\bf R4}
\begin{verbatim*}
int s1/0
 ip addr 66.51.46.4 255.255.255.248
 encapsulation frame-relay
 frame-relay map ip 66.51.46.6 406
 no frame-relay inverse-arp
 no shut
!
\end{verbatim*}
\item {\bf R6}
\begin{verbatim*}
int s1/0
 ip addr 66.51.46.6 255.255.255.248
 encapsulation frame-relay
 frame-relay map ip 66.51.46.4 604
 no frame-relay inverse-arp
 no shut
!
\end{verbatim*}
\item {\bf Capture}
\begin{verbatim*}
R6#show ip route
Gateway of last resort is not set

66.0.0.0/29 is subnetted, 1 subnets
C       66.51.46.0 is directly connected, Serial1/0
R6#show frame-relay map
Serial1/0 (up): ip 66.51.46.4 dlci 604(0x25C,0x94C0), static,
                CISCO, status defined, active

R4#show ip route
Gateway of last resort is 77.51.7.7 to network 0.0.0.0

66.0.0.0/29 is subnetted, 1 subnets
C       66.51.46.0 is directly connected, Serial1/0
77.0.0.0/24 is subnetted, 1 subnets
C       77.51.7.0 is directly connected, FastEthernet0/0
S*   0.0.0.0/0 [254/0] via 77.51.7.7

R4#show frame-relay map
Serial1/0 (up): ip 66.51.46.6 dlci 406(0x196,0x6460), static,
                CISCO, status defined, active
\end{verbatim*}
\end{itemize}
\subsection{Configuration of Frame Relay using point-to-point sub-interface of serial interface}
\begin{itemize}
\item {\bf R5}
\begin{verbatim*}
int s1/0
 encapsulation frame-relay
 no frame-relay inverse-arp
 no shut
!
int s1/0.51 point-to-point
 ip address 66.51.56.5 255.255.255.252
 frame-relay interface-dlci 516
!
\end{verbatim*}
\item {\bf R6}
\begin{verbatim*}
int s1/1
 encapsulation frame-relay
 no frame-relay inverse-arp
 no shut
!
int s1/1.51 point-to-point
 ip address 66.51.56.6 255.255.255.252
 frame-relay interface-dlci 615
!
\end{verbatim*}
\item {\bf Capture}
\begin{verbatim*}
R6#show ip route
Gateway of last resort is not set

     66.0.0.0/8 is variably subnetted, 2 subnets, 2 masks
C       66.51.56.4/30 is directly connected, Serial1/1.51
C       66.51.46.0/29 is directly connected, Serial1/0
R6#show frame-relay map
Serial1/0 (up): ip 66.51.46.4 dlci 604(0x25C,0x94C0), static,
              CISCO, status defined, active
Serial1/1.51 (up): point-to-point dlci, dlci 615(0x267,0x9870), broadcast
          status defined, active


R5#show ip route
Gateway of last resort is 77.51.7.7 to network 0.0.0.0

     66.0.0.0/30 is subnetted, 1 subnets
C       66.51.56.4 is directly connected, Serial1/0.51
     77.0.0.0/24 is subnetted, 1 subnets
C       77.51.7.0 is directly connected, FastEthernet0/0
S*   0.0.0.0/0 [254/0] via 77.51.7.7
R5#show frame-relay map
Serial1/0.51 (up): point-to-point dlci, dlci 516(0x204,0x8040), broadcast
          status defined, active
R6#ping 66.51.56.5

Type escape sequence to abort.
Sending 5, 100-byte ICMP Echos to 66.51.56.5, timeout is 2 seconds:
!!!!!
Success rate is 100 percent (5/5), round-trip min/avg/max = 32/68/96 ms
\end{verbatim*}
\end{itemize}
\subsection{Configuration of Frame Relay using point-to-multipoint sub-interface of serial interface}
\begin{itemize}
\item {\bf R2}
\begin{verbatim*}
int s1/0
 ip addr 66.52.236.2 255.255.255.248
 encapsulation frame-relay
 frame-relay map ip 66.52.236.6 216
 no frame-relay inverse-arp
 no shut
!
\end{verbatim*}
\item {\bf R3}
\begin{verbatim*}
int s1/0
 ip addr 66.52.236.3 255.255.255.248
 encapsulation frame-relay
 frame-relay map ip 66.52.236.6 316
 no frame-relay inverse-arp
 no shut
!
\end{verbatim*}
\item {\bf R6}
\begin{verbatim*}
int s1/1.52 multipoint
 ip addr 66.52.236.6 255.255.255.248
 frame-relay map ip 66.52.236.2 612
 frame-relay map ip 66.52.236.3 613
!
\end{verbatim*}
\item {\bf Capture}
\begin{verbatim*}
R2#show ip route
Gateway of last resort is 77.51.7.7 to network 0.0.0.0

     66.0.0.0/29 is subnetted, 1 subnets
C       66.52.236.0 is directly connected, Serial1/0
     77.0.0.0/24 is subnetted, 1 subnets
C       77.51.7.0 is directly connected, FastEthernet0/0
S*   0.0.0.0/0 [254/0] via 77.51.7.7
R2#show frame map
Serial1/0 (up): ip 66.52.236.6 dlci 216(0xD8,0x3480), static,
              CISCO, status defined, active

R3#show ip route
Gateway of last resort is 77.51.7.7 to network 0.0.0.0

     66.0.0.0/29 is subnetted, 1 subnets
C       66.52.236.0 is directly connected, Serial1/0
     77.0.0.0/24 is subnetted, 1 subnets
C       77.51.7.0 is directly connected, FastEthernet0/0
S*   0.0.0.0/0 [254/0] via 77.51.7.7
R3#show frame map
Serial1/0 (up): ip 66.52.236.6 dlci 316(0x13C,0x4CC0), static,
              CISCO, status defined, active

R6#show ip route
Gateway of last resort is not set

     66.0.0.0/8 is variably subnetted, 3 subnets, 2 masks
C       66.51.56.4/30 is directly connected, Serial1/1.51
C       66.51.46.0/29 is directly connected, Serial1/0
C       66.52.236.0/29 is directly connected, Serial1/1.52
R6#show frame map
Serial1/0 (up): ip 66.51.46.4 dlci 604(0x25C,0x94C0), static,
              CISCO, status defined, active
Serial1/1.52 (up): ip 66.52.236.2 dlci 612(0x264,0x9840), static,
              CISCO, status defined, active
Serial1/1.52 (up): ip 66.52.236.3 dlci 613(0x265,0x9850), static,
              CISCO, status defined, active
Serial1/1.51 (up): point-to-point dlci, dlci 615(0x267,0x9870), broadcast
          status defined, active
R6#ping 66.52.236.2

Type escape sequence to abort.
Sending 5, 100-byte ICMP Echos to 66.52.236.2, timeout is 2 seconds:
!!!!!
Success rate is 100 percent (5/5), round-trip min/avg/max = 28/64/104 ms
R6#ping 66.52.236.3

Type escape sequence to abort.
Sending 5, 100-byte ICMP Echos to 66.52.236.3, timeout is 2 seconds:
!!!!!
Success rate is 100 percent (5/5), round-trip min/avg/max = 8/56/136 ms
\end{verbatim*}
\end{itemize}
\section{Configuration of NAT}
\subsection{Verification of source IP address from TELNET}
\begin{verbatim*}
R2#telnet R8
Translating "R8"...domain server (177.51.7.8) [OK]
Trying R8.ie.cuhk.edu.hk (177.51.7.8)... Open


User Access Verification

Username: hello
Password:
R8>show user
Line       User       Host(s)              Idle       Location
0 con 0                idle                 02:24:09
*226 vty 0     hello      idle                 00:00:00 R2.ie.cuhk.edu.hk

Interface    User               Mode         Idle     Peer Address

R8>exit

[Connection to R8 closed by foreign host]
\end{verbatim*}
The source IP address of R2 when telnet to R8 is {\tt 77.51.7.2(R2.ie.cuhk.edu.hk)}.
\subsection{Configuration of PAT}
\begin{itemize}
\item {\bf R7}
\begin{verbatim*}
int f0/0
 ip nat inside
!
int f1/0
 ip nat outside
!
access-list 1 permit 77.51.7.0 0.0.0.255
ip nat inside source list 1 interface FastEthernet1/0 overload
\end{verbatim*}
\item {\bf Capture}
\begin{verbatim*}
R2#telnet R8
Translating "R8"...domain server (177.51.7.8) [OK]
Trying R8.ie.cuhk.edu.hk (177.51.7.8)... Open


User Access Verification

Username: hello
Password:
R8>show user
Line       User       Host(s)              Idle       Location
0 con 0                idle                 00:09:11
*226 vty 0     hello      idle                 00:00:00 R177.ie.cuhk.edu.hk

Interface    User               Mode         Idle     Peer Address

R8>exit

[Connection to R8 closed by foreign host]


R7#show ip nat trans
Pro Inside global      Inside local       Outside local      Outside global
udp 177.51.7.7:49997   77.51.7.2:49997    177.51.7.8:53      177.51.7.8:53
tcp 177.51.7.7:55229   77.51.7.2:55229    177.51.7.8:23      177.51.7.8:23
\end{verbatim*}
\end{itemize}
\subsection{Configuration of Static PAT}
\begin{itemize}
\item {\bf R7}
\begin{verbatim*}
ip nat inside source static tcp 77.51.7.3 23 177.51.7.7 3003 
ip nat inside source static tcp 77.51.7.4 23 177.51.7.7 3004
\end{verbatim*}
\item {\bf Verify and capture}
\begin{verbatim*}
R8#telnet R177 3003
Trying R177.ie.cuhk.edu.hk (177.51.7.7, 3003)... Open


User Access Verification

Username: hello
Password:
R3>show user
Line       User       Host(s)              Idle       Location
0 con 0                idle                 00:24:06
*226 vty 0     hello      idle                 00:00:00 R8.ie.cuhk.edu.hk

Interface    User               Mode         Idle     Peer Address

R3>exit

[Connection to R177 closed by foreign host]
R8#telnet R177 3004
Trying R177.ie.cuhk.edu.hk (177.51.7.7, 3004)... Open


User Access Verification

Username: hello
Password:
R4>show user
Line       User       Host(s)              Idle       Location
0 con 0                idle                 00:46:03
*226 vty 0     hello      idle                 00:00:00 R8.ie.cuhk.edu.hk

Interface    User               Mode         Idle     Peer Address

R4>exit

[Connection to R177 closed by foreign host]
R8#

R7#show ip nat trans
Pro Inside global      Inside local       Outside local      Outside global
tcp 177.51.7.7:3003    77.51.7.3:23       177.51.7.8:34046   177.51.7.8:34046
tcp 177.51.7.7:3003    77.51.7.3:23       ---                ---
tcp 177.51.7.7:3004    77.51.7.4:23       ---                ---
\end{verbatim*}
\end{itemize}
\subsection{Configuration of Static NAT}
\begin{itemize}
\item {\bf R7}
\begin{verbatim*}
ip nat inside source static 77.51.7.2 177.51.7.2
\end{verbatim*}
\item {\bf Verify and capture}
\begin{verbatim*}
R2# telnet R8
Translating "R8"...domain server (177.51.7.8) [OK]
Trying R8.ie.cuhk.edu.hk (177.51.7.8)... Open


User Access Verification

Username: hello
Password:
R8>show user
Line       User       Host(s)              Idle       Location
0 con 0                idle                 00:00:37
*226 vty 0     hello      idle                 00:00:00 177.51.7.2

Interface    User               Mode         Idle     Peer Address

R8>exit

[Connection to R8 closed by foreign host]

R8#telnet 177.51.7.2
Trying 177.51.7.2 ... Open


User Access Verification

Username: hello
Password:
R2>show user
Line       User       Host(s)              Idle       Location
0 con 0                R8                   00:00:18
*226 vty 0     hello      idle                 00:00:00 R8.ie.cuhk.edu.hk

Interface    User               Mode         Idle     Peer Address

R2>exit

[Connection to 177.51.7.2 closed by foreign host]



R7(config)#do show ip nat trans
Pro Inside global      Inside local       Outside local      Outside global
tcp 177.51.7.2:23      77.51.7.2:23       177.51.7.8:14858   177.51.7.8:14858
tcp 177.51.7.2:23      77.51.7.2:23       177.51.7.8:19347   177.51.7.8:19347
tcp 177.51.7.2:45372   77.51.7.2:45372    177.51.7.8:23      177.51.7.8:23
udp 177.51.7.2:54727   77.51.7.2:54727    177.51.7.8:53      177.51.7.8:53
--- 177.51.7.2         77.51.7.2          ---                ---
tcp 177.51.7.7:3003    77.51.7.3:23       ---                ---
tcp 177.51.7.7:3004    77.51.7.4:23       ---                ---
\end{verbatim*}
\end{itemize}
\section{Configuration of Network Security}
\begin{itemize}
\item {\bf R7}
\begin{verbatim*}
access-list 2 permit 177.51.7.8
access-list 2 permit 77.51.7.0 0.0.0.255
line vty 0 4
 access-class 2 in
!
\end{verbatim*}
\item {\bf Verify and capture}
\begin{verbatim*}
R2#telnet R77
Translating "R77"...domain server (177.51.7.8) [OK]
Trying R77.ie.cuhk.edu.hk (77.51.7.7)... Open


User Access Verification

Username: hello
Password:
R7>exit

[Connection to R77 closed by foreign host]

R8#telnet R177
Trying R177.ie.cuhk.edu.hk (177.51.7.7)... Open


User Access Verification

Username: hello
Password:
R7>exit

[Connection to R177 closed by foreign host]

R1# telnet R177
Translating "R177"...domain server (177.51.7.8) [OK]
Trying R177.ie.cuhk.edu.hk (177.51.7.7)...
% Connection refused by remote host
\end{verbatim*}
\end{itemize}
\section{Configuration of System logging}
\begin{itemize}
\item {\bf R7}
\begin{verbatim*}
logging facility sys9
logging 77.51.7.250
\end{verbatim*}
\item {\bf Verify and capture}
\begin{verbatim*}
R7#show logging
Syslog logging: enabled (11 messages dropped, 0 messages rate-limited,
                0 flushes, 0 overruns, xml disabled, filtering disabled)
    Console logging: level debugging, 21 messages logged, xml disabled,
                     filtering disabled
    Monitor logging: level debugging, 0 messages logged, xml disabled,
                     filtering disabled
    Buffer logging: disabled, xml disabled,
                    filtering disabled
    Logging Exception size (4096 bytes)
    Count and timestamp logging messages: disabled

No active filter modules.

    Trap logging: level informational, 25 message lines logged
        Logging to 77.51.7.250 (udp port 514, audit disabled, link up), 25 message lines logged, xml disabled,
               filtering disabled
\end{verbatim*}
\end{itemize}
\section{Configuration of SNMP support}
\begin{itemize}
\item {\bf R7}
\begin{verbatim*}
access-list 3 permit 77.51.7.3
access-list 3 permit 77.51.7.2
snmp-server community HELLOWORLD RO 3
\end{verbatim*}
\item {\bf Verify and capture}
\begin{verbatim*}
R7#show snmp community

Community name: ILMI
Community Index: cisco0
Community SecurityName: ILMI
storage-type: read-only  active


Community name: HELLOWORLD
Community Index: cisco1
Community SecurityName: HELLOWORLD
storage-type: nonvolatile        active access-list: 3
\end{verbatim*}
\end{itemize}
\newpage
\appendix
\section{Declaration}
I declare that the assignment here submitted is original except for source material explicitly acknowledged, and that the same or related material has not been previously submitted for another course. I also acknowledge that I am aware of University policy and regulations on honesty in academic work, and of the disciplinary guidelines and procedures applicable to breaches of such policy and regulations, as contained in the website \url{http://www.cuhk.edu.hk/policy/academichonesty/}
\vspace*{3cm}
\end{document}
